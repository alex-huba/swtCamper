% Author: Eugene Yip
% Last modified: 15 October 2019


% ----------------------------------------------------
% Settings common to all documents
\makeatletter
    \if@twoside%
        \usepackage[a4paper, margin = 2.3cm]{geometry}
    \else%
        \usepackage[a4paper, top = 2cm, bottom = 2.5cm, left = 3cm, right = 3cm, includehead, includefoot]{geometry}
    \fi%
\makeatother

\pdfminorversion=6

% Boolean variables to control the document style
\newif\ifsolutionmode	% To switch between blank answer mode or solution mode
\newif\ifshowmarkingbox	% To switch the marking box beside the allocated marks on or off

\usepackage{graphicx}

\newcommand\swtheader{
	\noindent\raisebox{-1.8ex}{\includegraphics[height=1cm]{\includedirectory/SWT-Logo}}
	\hfill\parbox{0.7\linewidth}{\centering \heading Lehrstuhl Softwaretechnik \& Programmiersprachen\\Fakult{\"a}t Wirtschaftsinformatik \& Angewandte Informatik}
	\hfill\raisebox{-4ex}{\includegraphics[height=1.8cm]{\includedirectory/UB-Logo}}
}

\newcommand\swtfullheader[1]{
	\swtheader
	\begin{center}
		{\large \heading #1\\[1ex]\courseacronym: \coursename}\\[3ex]
		{\heading \semesterlong}
	\end{center}
	\rule{\textwidth}{1pt}
	\vspace{0em}
}

\newcommand\swtcompactheader{
	\swtheader
	\begin{center}
		{\large \heading \courseacronym: \coursename}\\[3ex]
	\end{center}
	{\sffamily \lecturer \hfill \semesterlong}
	\newline
	\rule{\textwidth}{1pt}
	\vspace{0em}
}

\newcommand\genericheader[1]{
	\thispagestyle{footeronly}
	\clearpage
	\mbox{}\vspace{-1.5cm}

	\swtcompactheader

	\makemaintitle{#1}
}

% Global definition of headings
\usepackage{fancyhdr}
\setlength{\headheight}{14.5pt}
\pagestyle{fancy}
\lhead{\courseacronym (\semester)} \chead{} \rhead{\documentheader}
\lfoot{\issuedate}                 \cfoot{} \rfoot{\thepage~of~\pageref{LastPage}}

% Define a specific style for assignments and exercise sheets
\fancypagestyle{footeronly}{
	\lhead{}           \chead{} \rhead{}
	\renewcommand{\headrulewidth}{0pt}
	\lfoot{\issuedate} \cfoot{} \rfoot{\thepage~of~\pageref{LastPage}}
}

% Define a specific style for the proposal sheet of written exams
\fancypagestyle{proposalfooter}{
	\lhead{}           \chead{} \rhead{}
	\renewcommand{\headrulewidth}{0pt}
	\lfoot{\issuedate} \cfoot{} \rfoot{(\romannumeral\thepage)}
}

% Define a specific style for reports
\fancypagestyle{report}{
	\fancyhead{}
	\fancyhead[RE,LO]{\nouppercase\rightmark}
	\fancyhead[LE,RO]{\nouppercase\leftmark}
	
	\fancyfoot{}
	\fancyfoot[RE,LO]{\maintitle}
	\fancyfoot[LE,RO]{\thepage~of~\pageref{LastPage}}
}

% Copyright watermark
\usepackage{xifthen}
\newboolean{displaycopyright}
\setboolean{displaycopyright}{true}
\usepackage{atbegshi}
\AtBeginShipout{
	\ifthenelse{\boolean{displaycopyright}}{
		\AtBeginShipoutUpperLeft{
			\begin{tikzpicture}[remember picture,overlay]
				\draw (current page.south east) node[rotate=90,anchor=south west,inner sep=10pt] 
					{\color{black}\textnormal{\tiny\textrm{\textcopyright}~Lehrstuhl SWT \the\year{} (\author/\version). Die Weitergabe dieses Dokuments ist untersagt / The unauthorised distribution of this document is prohibited.}};
			\end{tikzpicture}
		}
	}{
		% Nothing
	}
}

\usepackage[utf8x]{inputenc}
\usepackage[T1]{fontenc}
\usepackage{textcomp}		% For straight quote marks in code listings
\usepackage{amsmath,amssymb,amsthm}
\usepackage{array}
\usepackage{forloop}
\usepackage{xspace}
\usepackage{graphicx}
\usepackage{pdfpages}
\usepackage{enumitem}
\usepackage{lastpage}
\usepackage{parskip}
\usepackage{listings}
\usepackage{marginnote}
\usepackage[innertopmargin = 1em, innerbottommargin = 1em, skipabove = 1em]{mdframed}
\usepackage[hypertexnames = false, hidelinks]{hyperref}

\usepackage{sectsty}
\allsectionsfont{\sffamily}
\makeatletter	% Adjust spacing after section and subsection titles
\renewcommand\section{\@startsection {section}{1}{\z@}%
      {-3.5ex \@plus -1ex \@minus -.2ex}% <beforeskip>
      {1ex \@plus.2ex}% <afterskip>
      {\normalfont\Large\bfseries\SS@sectfont}}
\renewcommand\subsection{\@startsection{subsection}{2}{\z@}%
      {-3.25ex\@plus -1ex \@minus -.2ex}% <beforeskip>
      {0.5ex \@plus .2ex}% <afterskip>
      {\normalfont\large\bfseries\SS@subsectfont}}
\makeatother


\renewcommand{\ttdefault}{lmtt}

\usepackage{tikz}
\usetikzlibrary{positioning, arrows, calc, shapes}


% Redefinition of marginnote to force margin notes to be on the right hand side.
% This is needed to keep the marking boxes next to the allocated marks.
% https://tex.stackexchange.com/a/69624
\makeatletter
\long\def\@mn@@@marginnote[#1]#2[#3]{%
  \begingroup
    \ifmmode\mn@strut\let\@tempa\mn@vadjust\else
      \if@inlabel\leavevmode\fi
      \ifhmode\mn@strut\let\@tempa\mn@vadjust\else\let\@tempa\mn@vlap\fi
    \fi
    \@tempa{%
      \vbox to\z@{%
        \vss
        \@mn@margintest
        \if@reversemargin\if@tempswa
            \@tempswafalse
          \else
            \@tempswatrue
        \fi\fi
          \rlap{%
            \ifx\@mn@currxpos\relax
              \kern\marginnoterightadjust
              \if@mn@verbose
                \PackageInfo{marginnote}{%
                  xpos not known,\MessageBreak
                  using \string\marginnoterightadjust}%
              \fi
            \else\ifx\@mn@currxpos\@empty
                \kern\marginnoterightadjust
                \if@mn@verbose
                  \PackageInfo{marginnote}{%
                    xpos not known,\MessageBreak
                    using \string\marginnoterightadjust}%
                \fi
              \else
                \if@mn@verbose
                  \PackageInfo{marginnote}{%
                    xpos seems to be \@mn@currxpos,\MessageBreak
                    \string\marginnoterightadjust
                    \space ignored}%
                \fi
                \begingroup
                  \setlength{\@tempdima}{\@mn@currxpos}%
                  \kern-\@tempdima
                  \if@twoside\ifodd\@mn@currpage\relax
                      \kern\oddsidemargin
                    \else
                      \kern\evensidemargin
                    \fi
                  \else
                    \kern\oddsidemargin
                  \fi
                  \kern 1in
                \endgroup
              \fi
            \fi
            \kern\marginnotetextwidth\kern\marginparsep
            \vbox to\z@{\kern\marginnotevadjust\kern #3
              \vbox to\z@{%
                \hsize\marginparwidth
                \linewidth\hsize
                \kern-\parskip
                \marginfont\raggedrightmarginnote\strut\hspace{\z@}%
                \ignorespaces#2\endgraf
                \vss}%
              \vss}%
          }%
      }%
    }%
  \endgroup
}
\makeatother


% ----------------------------------------------------
% Settings for question headings

% Heading font style
\newcommand\heading{\sffamily\bfseries}

% Empty box for writing in the achieved marks
\newcommand\markbox[1][]{%
	\ifshowmarkingbox{%
		\ifthenelse{\isempty{#1}}{%
			\marginnote{\tikz\draw (0,0) rectangle (1.2cm,1.2cm);}%
		}{%
			\marginnote{\tikz\draw[#1] (0,0) rectangle (1.5cm,1.5cm);}[-0.5cm]%
		}%
	}%
	\else%
		% Nothing
	\fi%
}

% Creating questions and sub-questions
% For example, Question x.y
% where x = \thequestion and y = \thesubquestion
\newcounter{question}
\renewcommand\thequestion{\arabic{question}}
\setcounter{question}{0}

\newcounter{subquestion}
\renewcommand\thesubquestion{\arabic{subquestion}}
\setcounter{subquestion}{0}

\makeatletter	% For getting a proper reference number that displays the question and subquestion numbers
	\renewcommand\p@subquestion{\thequestion.}
\makeatother

\newcommand\question[2][]{
	\refstepcounter{question}
	\ifthenelse{\isempty{#1}}{
		\clearpage{\Large \heading Question \thequestion: #2\markbox[double, double distance=1mm]}
	}{
		\clearpage{\Large \heading Question \thequestion: #2 \hfill [#1m]\markbox[double, double distance=1mm]}
		
		% The new line immediately above allows the marks to appear flush against the right margin!
	}
	\setcounter{subquestion}{0}
	
	\newcounter{questiontotalmarks\arabic{question}}
	\setcounter{questiontotalmarks\arabic{question}}{0}
}

% Creating sub-questions
\newcommand\subquestion[1][]{
	\refstepcounter{subquestion}
	\par\vspace{1em}
	\ifthenelse{\isempty{#1}}{
		{\heading Question \thequestion.\thesubquestion}
	}{
		{\heading Question \thequestion.\thesubquestion \hfill [#1m]\markbox}

		% The new line immediately above allows the marks to appear flush against the right text area!
	}

	\addtocounter{questiontotalmarks\arabic{question}}{#1}
}

% Creating points/marks
\newcommand\pointsunbold[1]{{\sffamily [#1m]\xspace}}	% Needed when a breakdown of points is needed for a large question
\newcommand\points[1]{{\heading \pointsunbold{#1}}}


% Print out the marks
\newcommand\printpoints[1]{
	\newcounter{totalmarks}
	\setcounter{totalmarks}{0}
	\newcounter{questionnumber}
	\forLoop[1]{1}{\value{question}}{questionnumber}{
		\typeout{#1Question \thequestionnumber: \arabic{questiontotalmarks\thequestionnumber} marks}
		\addtocounter{totalmarks}{\value{questiontotalmarks\thequestionnumber}}
	}
	\typeout{#1Total: \arabic{totalmarks} marks}
}

% ----------------------------------------------------
% Settings for exercise headings

\newcounter{exercisenumber}
\renewcommand\theexercisenumber{\exercisesheetnumber.\arabic{exercisenumber}}
\setcounter{exercisenumber}{0}

\newcounter{subexercisenumber}
\renewcommand\thesubexercisenumber{\arabic{subexercisenumber}}
\setcounter{subexercisenumber}{0}

\makeatletter	% For getting a proper reference number that displays the exercise and subexercise numbers
	\renewcommand\p@subexercisenumber{\theexercisenumber.}
\makeatother

\newcommand{\exercise}[1]{
	\refstepcounter{exercisenumber}
	\pagebreak[0]	% Suggest to LaTeX to break a page if the end of the previous exercise flows to a new page.
	\par\vspace{2em}	% Vertical space is ignored at the start of a new page with the par command.
	{\Large \heading \theexercisenumber \hspace{1ex} #1}
	\setcounter{subexercisenumber}{0}
}

% Creating sub-exercises
\newcommand\subexercise[1][]{
	\refstepcounter{subexercisenumber}
	\pagebreak[0]
	\par\vspace{1em}
	{\heading Exercise \theexercisenumber.\thesubexercisenumber \hspace{1ex}}
}


% ----------------------------------------------------
% Settings mostly relevant to written exams 

\newcount{\gridcellcount}
\gridcellcount = 32
\newlength{\areawidth}
\newlength{\areaheight}
\newlength{\gridcellsize}
\newcommand\gridarea[1][]{% Draws a gridded area for written answers. Optional argument for the height of the gridded area. Fills in the remaining page by default.
	\vspace{1em}
	\setlength\areawidth{\textwidth - 8\pgflinewidth}
	\setlength\gridcellsize{\areawidth / \gridcellcount}
	\setlength\areaheight{\dimexpr \pagegoal - \pagetotal - 2\baselineskip}	% (Height of page body) - (Height of page body that has been used so far) - 2*(Height between two paragraphs)

	% Convert the grid cell size into a dimensionless value
	\newcount\gridcellvalue
	\gridcellvalue = \gridcellsize		

	% Convert the area height into a dimensionless value
	\newcount\areaheightvalue
	\ifthenelse{\isempty{#1}}{
		\areaheightvalue = \dimexpr\areaheight
	}{
		\areaheightvalue = \dimexpr#1			
	}

	% Convert the height into the number grid cells
	\divide\areaheightvalue by\gridcellvalue	

	\begin{center}	% Centering needed to prevent overfull or underfull warnings
		\tikz\draw[step = \gridcellsize, color = gray!50] (0,0) grid (\areawidth, \the\areaheightvalue\gridcellsize);
	\end{center}
}

\newcommand\gridpage{		% Fills in the rest of the page with grids for written answers
	\clearpage
	\textbf{(Additional space; carefully label the question you are answering.)}
	
	\gridarea
	\clearpage
}

\newcommand\xarea[1][]{	% Crosses out the rest of the page to prevent written answers
	\ifsolutionmode \else
		\ifthenelse{\isempty{#1}}{
			\setlength\areaheight{\dimexpr \pagegoal - \pagetotal - 2\baselineskip}	% (Height of page body) - (Height of page body that has been used so far) - 2*(Height between two paragraphs)
		}{
			\areaheight=\dimexpr#1			
		}
		\tikz[remember picture,overlay]
			\draw[gray](0,0) -- (\textwidth,-\areaheight) 
					   (\textwidth,0) -- (0,-\areaheight) 
					   (\textwidth/2,-\areaheight/2) node[font=\LARGE\bfseries]{Do not write in this space.};
		\clearpage
	\fi
}

\newcommand\xpage{ % Crosses out an entire page to prevent written answers
	\ifsolutionmode \else
		\clearpage
		\setboolean{displaycopyright}{false}
		\tikz[remember picture,overlay]
			\draw[gray](current page.north west) -- (current page.south east) 
					   (current page.south west) -- (current page.north east) 
					   (current page.center) node[font=\LARGE\bfseries]{Do not write on this page.};
	\fi
	\clearpage
	\setboolean{displaycopyright}{true}
}

% Project proposal sheet
\newcommand\proposalsheet[2]{
	% ----------------------------------------------------
	% Front side of the proposal sheet
	
	\thispagestyle{proposalfooter}
	\mbox{}\vspace{-1.5cm}

	\swtfullheader{Klausur zum Modul}

	\noindent
	\textbf{Name:}~~\underline{\hspace{7.5cm}}\hfill 
	\textbf{Matrikelnr.:}~~\underline{\hspace{4cm}}

	\vspace{1em}

	\noindent
	\textbf{Studiengang:}~~\underline{\hspace{6.26cm}}\hfill 
	\textbf{Fachsemester:}~~\underline{\hspace{3.64cm}}
	
	\vspace{2cm}

	\textbf{Carefully read the following proposal, which concerns most exam questions.}

	\textbf{This sheet is to be handed in with the questions \& answers' part of the exam.}

	\vspace{1cm}
	
	% Proposal is inserted here
	#1
	
	\vfill

	\textbf{Note: Where there is a lack of detail in the proposal above,
	you are permitted to make assumptions. However, you shall always clearly
	state those assumptions~in your answers.}
	
	\clearpage
	
	% ----------------------------------------------------
	% Back side of the proposal sheet
	
	\thispagestyle{proposalfooter}
	
	% Contents of the second page are inserted here
	#2
}

% Front and back pages
\newcommand\examfrontpage[1]{
	\ifsolutionmode \else 
		\setboolean{displaycopyright}{false}
		\includepdf[pages=1]{\includedirectory/Umschlag-SS16.pdf}
		\setboolean{displaycopyright}{true}
	\fi
	
	\thispagestyle{empty}
	\setcounter{page}{0}
	\clearpage
	\mbox{}\vspace{-1.5cm}

	\swtfullheader{Klausur zum Modul}
	
	% Insert contents of the front page here
	#1
}

\newcommand\exambackpage{
	\ifsolutionmode \else
		\pagestyle{empty}
		\setboolean{displaycopyright}{false}
		\includepdf[pages={2,3}]{\includedirectory/Umschlag-SS16.pdf}
		\addtocounter{page}{-2}
		\setboolean{displaycopyright}{true}
	\fi
	
	% Log messages
	\typeout{}
	\typeout{-----------------------------------------------------}

	\printpoints{ }
	\typeout{}
	
	\typeout{ Total number of exam pages must be divisible by 4!}
	\typeout{-----------------------------------------------------}
}


% ----------------------------------------------------
% Settings mostly relevant to written test exams 

\newcommand\testexamfrontpage[3]{
	\thispagestyle{empty}
	\setcounter{page}{0}
	\mbox{}\vspace{-1.5cm}
	
	\swtfullheader{#1}
	
	\noindent
	\textbf{Name:}~~\underline{\hspace{7.5cm}}\hfill 
	\textbf{Matrikelnr.:}~~\underline{\hspace{4cm}}

	\vspace{1em}

	\noindent
	\textbf{Studiengang:}~~\underline{\hspace{6.26cm}}\hfill 
	\textbf{Fachsemester:}~~\underline{\hspace{3.64cm}}
	
	\vspace{2cm}
	
	% Insert contents of the front page here
	#2
	
	\clearpage

	\textbf{Note: Carefully read the following proposal, which concerns most test exam questions.}
	
	\textbf{Where there is a lack of detail in the proposal above,
	you are permitted to make assumptions. However, you shall always clearly
	state those assumptions~in your answers.}
	
	\vspace{4em}
	
	% Insert project brief here
	#3

	\clearpage
}


% ----------------------------------------------------
% Settings mostly relevant to written assignments and exercise sheets

\newcommand\makemaintitle[1]{
	\begin{center}
		\Large \sffamily \textbf{#1}\\[1.5ex]
	\end{center}
}

\newcommand\makesubtitle[1]{
	\begin{center}
		\Large \sffamily #1\\[1.5ex]
	\end{center}
}

\newcommand\assignmentfrontpage[3][]{
	\thispagestyle{footeronly}
	\mbox{}\vspace{-1.5cm}

	\swtcompactheader
	
	% Main document title is inserted here
	\makemaintitle{#2}
	\ifthenelse{\isempty{#1}}{
	}{
		\makesubtitle{#1}
	}

	\vspace{1em}
	
	\begin{center}
		\textbf{\Large \sffamily Strict deadline: \deadlinedate at \deadlinetime}
	\end{center}

	% Rest of the front page is inserted here
	#3
	
	\clearpage
}

\newcommand\declaration[1]{
	\thispagestyle{footeronly}
	\mbox{}\vspace{-1.5cm}

	\swtcompactheader
	\makemaintitle{#1}

	\vspace{1em}

	\section*{Ehrenwörtliche Erklärung}
	\label{sec:erklaerung}
	Ich habe die vorliegende Hausarbeit einschließlich der digitalen Abgabe
	selbständig verfasst und keine anderen als die angegebenen Quellen und
	Hilfsmittel benutzt.

	\vspace{2cm}

	\hrule
	Matrikelnummer \hspace{4cm} Name

	\vspace{2cm}

	\hrule
	Ort/Datum \hspace{4.9cm} Unterschrift
	
	\vfill
	
	\hrule
	Gesamtnote \hspace{4.9cm} Datum

	\vspace{2cm}

	\hrule
	Prüfer \hspace{5.9cm} Unterschrift
	
	\vspace{1cm}
	
	% Log messages
	\typeout{}
	\typeout{-----------------------------------------------------}
	\printpoints{ }
	\typeout{}
	\typeout{ Student declaration page must be single sided!}
	\typeout{-----------------------------------------------------}
}


% ----------------------------------------------------
% Settings mostly relevant to exercise sheets

\newcommand\practicalheader[1]{
	\thispagestyle{footeronly}
	\clearpage
	\mbox{}\vspace{-1.5cm}

	\swtcompactheader

	\makemaintitle{#1 Sheet \exercisesheetnumber: \documentheader}
}

\newcommand\exerciseheader{\practicalheader{Exercise}}
\newcommand\supplementaryheader{\practicalheader{Supplementary}}


% ----------------------------------------------------
% Settings mostly relevant to reports

\newcommand\reportfrontpage[1]{
	\setboolean{displaycopyright}{false}

	\thispagestyle{empty}
	\clearpage
	
	\mbox{}\vspace{-1.5cm}

	\swtheader

	\vspace{2cm}

	\makemaintitle{\maintitle}
	\makesubtitle{\subtitle}
	
	\vspace{2cm}
		
	\begin{center}
		\sffamily
		
		\preparedfor.
	
		By \reportauthor, \affiliation.
	
		\vspace{1em}
	
		Version: \publishdate
	
		\vfill
	
		\begin{minipage}{11cm}
			\textbf{Abstract:}
			\setlength{\parindent}{1pc}
			 #1
		\end{minipage}

		\vfill
	\end{center}
	
	
	% Verify that abstract is not too long
	\ifnum \value{page}>1
		\errhelp{Please ensure that the abstract remains short enough to remain on the title page}
		\errmessage{Abstract too long}
	\else
	\fi
	
	\clearpage
	\pagestyle{report}
}
	
\newcommand\studentreportfrontpagetop[5]{
	\newgeometry{margin = 2.3 cm}			% Front page has narrower margins
	\setboolean{displaycopyright}{false}

	\pagestyle{empty}
	\clearpage
	
	\mbox{}\vspace{-1.5cm}

	\swtheader

	\vspace{2cm}

	\makemaintitle{#1}
	
	\vspace{1em}
	
	\makemaintitle{#2}
	
	\vspace{1em}
	
	\makesubtitle{#3}
	\makesubtitle{#4}
	
	\vspace{2cm}
	
	\begin{center}
		\sffamily
		
		#5
	
		\vfill
	\end{center}
	
	% Restore the page margins 
	\restoregeometry
}

\newcommand\studentdetails[4]{
	\textbf{#1}						\\
	Student Number: #2				\\
	Degree Course/Semester: #3/#4
	
	\vspace{1em}
}

\newcommand\studentsignature[1]{
	\vspace{0.5cm}

	\begin{mdframed}
		\vspace{0.5em}
		\hspace{5.9cm} \emph{\large #1}
		\vspace{0.5em}
		\hrule
		Matrikelnummer \hspace{3cm} Name

		\vspace{1.4cm}

		\hrule
		Ort, Datum \hspace{3.77cm} Unterschrift
	\end{mdframed}
}

% ----------------------------------------------------
% Settings for code formating

\usepackage{xcolor}
\definecolor{darkviolet}{rgb}{0.5,0,0.4}
\definecolor{darkred}{rgb}{0.6,0.0,0.0}
\definecolor{darkgreen}{rgb}{0,0.4,0.2}
\definecolor{darkblue}{rgb}{0.1,0.1,0.9}
\definecolor{darkgrey}{rgb}{0.5,0.5,0.5}
\definecolor{lightblue}{rgb}{0.4,0.4,1}
\definecolor{mauve}{rgb}{0.58,0,0.82}

\lstdefinestyle{eclipsish}{
	basicstyle=\normalsize\ttfamily,
	emphstyle=\color{red}\bfseries,
	keywordstyle=\color{darkgreen}\bfseries,
	keywordstyle=[2]\color{darkviolet}\bfseries,
	commentstyle=\color{darkgrey},
	stringstyle=\color{darkblue},
	numberstyle=\color{darkgrey}\ttfamily\tiny,
	emphstyle=\color{red},
	morecomment=[s][\color{lightblue}]{/**}{*/},
	showstringspaces=false,
	numbers=left,
	numbersep=5pt,
	captionpos=b,                   % sets the caption-position to bottom
	xleftmargin=3.5ex,
	xrightmargin=2.4ex,
	breakindent=3ex,
	breakautoindent,
	breakatwhitespace=false,        % sets if automatic breaks should only happen at whitespace
	escapeinside={(*@}{@*)},
	mathescape=true,
	breaklines=true,
	tabsize=4,
	upquote=true
}

\lstdefinestyle{cpp}{
	numbers=none,	%left
	stringstyle=\color{red},
	keywordstyle=\color{darkred},
	commentstyle=\color{darkgreen},
	morekeywords={}% list your attributes here
}

\lstdefinelanguage{WHILE}[]{C}{
	morekeywords=[2]{skip,i,abort,par},
	alsoletter={^},
	morekeywords= {begin,end,fi,then,protect,repeat,do,od,until,end}
}

\lstdefinelanguage{SMV}[]{C}{
	morekeywords=[2]{:=,FALSE,TRUE},
	alsoletter={^},
	morekeywords= {MODULE,VAR,ASSIGN,DEFINE,case,esac,init,next},
	morecomment=[l]{--},
}

\lstdefinelanguage{XML}{
	morestring=[s][\color{mauve}]{"}{"},
	morestring=[s][\color{black}]{>}{<},
	morecomment=[s]{<?}{?>},
	morecomment=[s][\color{darkgreen}]{<!--}{-->},
	stringstyle=\color{black},
	identifierstyle=\color{lightblue},
	keywordstyle=\color{red},
	morekeywords={}% list your attributes here
}

\lstset{language=SMV,style=eclipsish}
