To minimize any misunderstandings between team members, a Glossary (\nameref{fig:glossary_1} \& \nameref{fig:glossary_2}) containing technical and project specific terms was initiated.
Next, under consultation of the Project Brief and the FSE-B slides on Project Blastoffs, we devised a \nameref{fig:context_work_diagram}. Here we show the three user types/stakeholders Renter, Provider, Operator as well as the data and information flows to and from the SWTcamper application.
A more comprehensive view of the stakeholders of the SWTcamper application can be seen on the \nameref{fig:stakeholder_map}. In addition to the prospective users of the application, we show the core development team, the client (SWT chair), the customer (Aracom IT Services AG), an internal consultant (Ms. Fradtschuk), some possible negative stakeholders and a regulator located in the wider environment.
As for \nameref{fig:risks}, we identified five possible scenarios, their risk type, probability, effect and mitigation strategy. By doing this, we can assure an uninterrupted workflow (to a reasonable degree).
We acquired solution, project and process constraints from the project and development brief. These give us a rough framework for the project and the possibility to talk about the technologies to be used, the timeframe available and how to conduct the project.

\begin{table}[h]
    \centering
    \caption{Contraints}
    \begin{tabular}{|p{4cm}|p{4cm}|p{4cm}|}
        \hline
        \textbf{Solution constraints} & \textbf{Project constraints} & \textbf{Process constraints} \\ \hline
        The SWTcamper application shall establish a web portal & The portal's prototype shall take at most 4 Sprints to build & A modern Java prototype of the portal shall be implemented \\ \hline
        The portal's prototype shall run on a single-user PC with a JavaFX user interface &  & The tool Git shall be used for version control via a GitLab repository \\ \hline
        The portal's prototype shall interact with a MariaDB database &  & The development shall follow SCRUM with a sprint length of 2 weeks \\ \hline
        An administration section shall be developed for the portal's operator &  &  \\ \hline
    \end{tabular}
\end{table}

For the purpose of getting a first idea of how the application could look and have something tangible to discuss with the customer and client, we sketched an \nameref{fig:gui_sketch}.
To establish an overview and common understanding of the SWTcamper application, we created a  \nameref{fig:use-case-diagram}, a preliminary high-level \nameref{fig:architecture_early_draft}, a \nameref{fig:er-diagram_draft} and a preliminary \nameref{fig:class_diagram_old}.