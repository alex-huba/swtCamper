\section{Requirements}
\label{sec:requirements}

\paragraph{Primary textual contributors:}
\mbox{}\\\emph{Dehom Melissa Pereira Gnassingbe, Patrick Willibald Haas}

\emph{Document and analyze the software's functional requirements, 
non-functional requirements and development constraints. In particular, state 
whether a requirement is derived from the project brief, is an assumption made 
by your team, or has been added by the client. You may apply any documentation 
and analysis technique taught in module SWT-FSE-B or from the requirements 
engineering literature, including techniques based on user stories, use cases 
and prototyping. Properly reference and justify all employed techniques.}


The following lists all requirements that had to be implemented for the SWTCamper software.
Our requirements engineering process contained the requirements' elicitation, requirements analysis and negotiation, requirements documentation and requirements validation.
Exactly as taught in the SWT-FSE-B module.
The different stages of the requirements engineering process were repeated in every sprint, sometimes several times.
In the blastoff-phase we derived all the constraints, functional and non-functional requirements from the project brief and also added some requirements we thought of while brainstorming in the team.
While working on the different user stories and updating the acceptance criteria, we also derived some more requirements.
But our main elicitation technique was interviewing.
Most of the requirements were gathered by interviewing the customer in the PO Meetings.
Another helpful source of requirements was the bi-weekly Sprint Review-Meetings, where we could ask the customer to refine the requirements.
During our development process, especially in the implementation part, there often came up changing requirements.
We validated those changing or new requirements with the customers' feedback in the review meetings.
% We also established some requirements that came up with during the implementation by getting feedback from the customer.
The customer also recommended us to create a prototype, so that we could more easily validate the requirements with them.
Indeed, the prototype allowed us to see if we understand the initial requirements well, and if there was a need to define more others.
Also, a lot of requirements (or their priority) changed during the sprint.
Therefore it was very important to keep asking questions and try to refine the requirements regularly.
On basis of the example project "xTASKS", the requirements were grouped by the different main functionalities of the project:
Register and login, Offering Campers, Search and filtering, Booking process, Administration section, User Management % and General.
The sources will be noted in brackets at the end of the requirement description.
It can either be the Project Brief, assumptions made by the team (in the following: team) or requirements of the customer.
If there is both the team and the customer noted, it means that it was an assumption made by the team which the customer saw as an important requirement, too, or that the customer detailed or changed the requirement more.

\subsection{Functional Requirements}

A lot of the functional requirements were changed or refined several times.
The following requirements represent the final ones.
If there has been a lot of changes and refining, it will be specified in a paragraph underneath the description.

\paragraph{RQ1: Registration and login}
\begin{itemize}
    \item The system shall provide means for a user to register to the portal by supplying user information, containing:  (Project Brief, Team)
        \subitem a unique username,
        \subitem a unique e-mail address,
        \subitem a password,
        \subitem name, surname and phone number.
    \item The user information shall only be accepted if they are valid: (Team)
        \subitem username and password with at least 5 characters
        \subitem name and surname without special characters.
    \item The user shall not be able to register without filling in the mandatory information fields. (Team, Customer)
    \item Registered users shall be saved into the database. (Team)
    \item The user shall be able to login to the portal by supplying the unique username and password. (Project Brief, Team, Customer)
    \item It shall not be possible for a user to gain access to the portal with the wrong username and password. (Team)
    \item It shall also not be possible for a user to use the portal's functionality without being logged in. (Team)
    \item Users shall be able to change their password by supplying their unique username and unique email address. (Team, Customer)
    \item OPTIONAL: The users passwords shall be encrypted. (Team, Customer)
    \item In addition to the home, offer search, login/logout and FAQ view, a renter shall only see the renter views: (Customer (Prototyping))
        \subitem Active bookings view
        \subitem Deal history view
    \item In addition to the home, offer search, login/logout and FAQ view, a provider shall only see the provider views: (Customer (Prototyping))
        \subitem Create offer view
        \subitem Active bookings view
        \subitem Deal history view
        \subitem Exclude and report user view
    \item In addition to the home, offer search, login/logout and FAQ view, an operator shall see all the views: (Customer (Prototyping))
        \subitem Create offer view
        \subitem Active bookings view
        \subitem Deal history view
        \subitem Exclude and report user view
        \subitem Accept / deny provider view
        \subitem Administation dashboard view
    \item It shall not be possible for providers to use providers functionality if they are not enabled by the operator yet.
\end{itemize}


\paragraph{RQ2: Camper van offering}
\begin{itemize}
    \item The system shall make it possible to create, edit and delete an offer. (Project Brief)
    \item The system shall make it possible to indicate / upload the following information about the vehicle: (Project Brief)
        \subitem Pictures
        \subitem Vehicle features
        \subitem Availability dates
        \subitem Rental conditions
        \subitem Particularities
    \item It shall not be possible to create an offer without filling in the mandatory information fields. (Team, Customer)
    \item When creating or updating an offer, the offer and its vehicle shall be saved into the database. (Team)
    \item In the offer creation process the provider shall be able to configure rental conditions. (Customer)
    \item The system shall provide means for a provider to access a list of their own offers. (Team)
    \item The list of offers shall contain a thumbnail for each offer. (Customer)
\end{itemize}

\paragraph{RQ3: Search and filter offers}
\begin{itemize}
    \item The system shall provide users search and filtering options when browsing for camper vans on offer. (Project Brief)
    \item It shall be able to search and filter offers regarding: (Project Brief, Customer)
        \subitem vehicle features,
        \subitem availability,
        \subitem rental costs,
    \item Inactive offers shall not be returned as a search result. (Team)
\end{itemize}

\paragraph{RQ4: Booking process}
\begin{itemize}
    \item Users shall be able to submit a booking request for an active offer.
    \item The request in the providers perspective shall contain information as: (Project Brief)
        \subitem booking dates
        \subitem renter's name
        \subitem rating
        \subitem total price
    \item The portal shall display a booking confirmation after a successful booking request. (Team)
    \item The provider shall receive a notification after a renter requested to book their camper van. (Customer)
    \item The provider shall also be able to accept or decline the booking request. (Customer)
    \item The renter shall get a booking confirmation after their booking request was accepted. (Team, Customer)
    \item After the booking got accepted by the provider, it shall be listed in both the renters and the providers booking history. (Team)
    \item A provider shall be able to cancel a booking in any stage of the booking process. (Customer)
    \item Users shall see their personal booking history, including all bookings they ever got accepted (renter) or ever accepted (provider).  (Team)
    \item A requested booking shall be saved into the database. (Team)
\end{itemize}

\paragraph{RQ5: Administration section}
\begin{itemize}
    \item The operator shall be able to configure the following options via an administration section:
        \subitem vehicle features
        \subitem number of promoted offers shown on one search result page
    \item
\end{itemize}

At the beginning, the customer said that it would be necessary for the operator to also configure the rental conditions.
After one or two iteration this requirement changed, and the customer preferred to let the provider configure the rental conditions by themselves, so that the provider would be more flexible to indicate their own needs.
It came up in a PO meeting.
This is a perfect example for a dynamic requirements change.
Also, the configuration of the vehicle features was one requirement that we couldn't make on time.
It had a lower priority, as the customer said it would not be mandatory, but they still wanted to have it.
We communicated it to the customer on time, and they were okay with us not implementing it.
Regarding the number of promoted offers shown on one search result page, we also discussed with the customer our changing requirement.
The reason why the customer wanted to have this feature, is because they wanted to prevent a search result page being filled with only promoted offers on the beginning if the position of the offer is being changed.
We argumented, that this problem would be resolved by not changing the position of a promoted offer and only highlighting it visually.
Because in later releases providers might want to pay to promote their offers, this solution pleased the customer.


\begin{itemize}
    \item The operator shall be able to block or unblock users. (Project Brief)
    \item Providers and operators shall be able to report a user. (Customer)
    \item The operator shall get a notification if a provider registered and accept or decline it. (Customer)
    \item It shall be possible for operators to influence search results. They shall be able to do so by highlighting an existing offer via a 'promote offer' button.
    \item Promoted offers shall be saved into the database.
    \item The operator shall be able to see a history / tracker of every activity in the SWTcamper via a log. (Team, Customer)
\end{itemize}


\paragraph{RQ6: User management}
\begin{itemize}
    \item The system shall differentiate between the providers' user role, the renters' user role and the operators user role. (Project Brief, Customer)
    \item Providers and operators shall be able to report a user. (Customer)
    \item A blocked user shall not be able to make any more deals. (Customer)
    \item A provider shall be able to exclude a renter from their offers, if the renter is not already yet excluded by them. They shall also be able to revoke the exclusion. (Project Brief, Team)
    \item Excluded renters shall not have the possibility to see the any offer of the provider who excluded them. (Team)
\end{itemize}

We also implemented a FAQ, as it was part of our prototype which we showed to the customer in the beginning.
The customer was very pleased to see, that we sticked to our prototype and even implemented an optional thing.
Of course, there would have been other open requirements with higher priority, as the operators' configuration of options.
But in terms of complexity and time required, the FAQ requirement was easier and quicker to implement in the last sprint.

\paragraph{Optional, RQ6: FAQ}
\begin{itemize}
    \item  The system shall provide means for (unregistered) users to read a FAQ. (Team)
    \item The FAQ section shall be accessible from the main menu. (Team)
    \item The FAQ section shall contain all neccessary information about renting and offering vehicles. (Team)
\end{itemize}

\subsection{Non-Functional Requirements}
Many of the non-functional requirements were gathered by using prototyping as an elicitation, and also validation, technique.
The prototype allowed us to get very specific requirement refinements from the customer, especially in the review meetings.
This made the elicitation process to collect non-functional requirements much easier.

\begin{itemize}
    \item The applications' design shall be leaned on the design of a web platform. (Customer)
    \item The system shall be easy to use and easy to administrate. (Project Brief)
    \item Potential renters shall have a positive experience throughout their offer browsing. (Project Brief)
    \item Providers shall be able to professionally advertise their vehicles. (Project Brief)
    \item The menu navigation shall be permanent / globally accessible and contain quick links. (Customer)
    \item The configuration of options shall be dynamic. (Customer)
    \item The layout of the application shall be responsive. (Customer)
    \item The offer list shall have a card layout. (Customer)
    \item The returned search list shall be small and clearly arranged. (Project Brief)
    \item OPTIONAL: The returned search list shall also be pageable. (Customer)
\end{itemize}


\subsection{Constraints}
To further specify our product we listed the constraints of the software.
\subsubsection{Solution Constraints}
\begin{itemize}
    \item The portal's prototype shall run on a single-user PC with a JavaFX user interface. (Source: Client/Project Brief)
    \item The portal's prototype shall interact with a MariaDB database. (Source: Client/Project Brief)
    \item An administration section shall be developed for the portal's operator. (Source: Client/Project Brief)
\end{itemize}
\subsubsection{Project Constraints}
\begin{itemize}
    \item The portal's prototype shall be developed during 5 sprints. (Source: Client/Project Brief)
    \item The project shall be handed in on February, 9th at 12pm. (Source: Client/Project Brief)
\end{itemize}
\subsubsection{Process Constraints}
\begin{itemize}
    \item For version control the tool Git shall be used via GitLab repository. (Source: Client/Project Brief)
    \item For documenting the source code JavaDoc shall be used. (Source: Client/Project Brief)
\end{itemize}