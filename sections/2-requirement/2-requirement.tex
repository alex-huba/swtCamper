\section{Requirement}
\label{sec:requirements}

\paragraph{Primary textual contributors.}
\mbox{}\\\emph{Dehom Melissa Pereira Gnassingbe, Patrick Willibald Haas}

\emph{Document and analyze the software's functional requirements, 
non-functional requirements and development constraints. In particular, state 
whether a requirement is derived from the project brief, is an assumption made 
by your team, or has been added by the client. You may apply any documentation 
and analysis technique taught in module SWT-FSE-B or from the requirements 
engineering literature, including techniques based on user stories, use cases 
and prototyping. Properly reference and justify all employed techniques.}


The following describes a list of requirements that shall be implemented by the SWTCamper software.
In the blastoff-phase we derived all the functional and non-functional requirements from the project brief and also added some requirements we thought of by brainstorming in the group.
Additional requirements were gathered by interviewing the customer in the PO Meetings.
Another helpful source of requirements is the bi-weekly Sprint Review-Meetings, where we could ask the customer for specific details on features.
Finally, we also established some requirements we came up with during the implementation by getting feedback from the customer.
On basis of the example project "xTASKS", the requirements were grouped by the different main functionalities of the project: Register and login, Offering Campers, Search and filtering, Booking process, Administration section, User Management and General.

\subsection{Functional Requirements}
\subsection{Non-Functional Requirements}
\subsection{Constraints}
\subsubsection{Solution Constraints}
\subsubsection{Project Constraints}
\subsubsection{Process Constraints}