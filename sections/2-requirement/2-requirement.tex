\section{Requirements}
\label{sec:requirements}

\paragraph{Primary textual contributors.}
\mbox{}\\\emph{Dehom Melissa Pereira Gnassingbe, Patrick Willibald Haas}

\emph{Document and analyze the software's functional requirements, 
non-functional requirements and development constraints. In particular, state 
whether a requirement is derived from the project brief, is an assumption made 
by your team, or has been added by the client. You may apply any documentation 
and analysis technique taught in module SWT-FSE-B or from the requirements 
engineering literature, including techniques based on user stories, use cases 
and prototyping. Properly reference and justify all employed techniques.}


The following lists all requirements that had to be implemented for the SWTCamper software.
Our requirements engineering process contained the requirements' elicitation, requirements analysis and negotiation, requirements documentation and requirements validation.
Exactly as taught in the SWT-FSE-B module.
The different stages of the requirements engineering process were repeated in every sprint, sometimes several times.
In the blastoff-phase we derived all the constraints, functional and non-functional requirements from the project brief and also added some requirements we thought of while brainstorming in the team.
While working on the different user stories and updating the acceptance criteria, we also derived some more requirements.
But our main elicitation technique was interviewing.
Most of the requirements were gathered by interviewing the customer in the PO Meetings.
Another helpful source of requirements was the bi-weekly Sprint Review-Meetings, where we could ask the customer to refine the requirements.
During our development process, especially in the implementation part, there often came up changing requirements.
We validated those changing or new requirements with the customers' feedback in the review meetings.
% We also established some requirements that came up with during the implementation by getting feedback from the customer.
The customer also recommended us to create a prototype, so that we could more easily validate the requirements with them.
Indeed, the prototype allowed us to see if we understand the initial requirements well, and if there was a need to define more others.
Also, a lot of requirements (or their priority) changed during the sprint.
Therefore it was very important to keep asking questions and try to refine the requirements regularly.
On basis of the example project "xTASKS", the requirements were grouped by the different main functionalities of the project:
Register and login, Offering Campers, Search and filtering, Booking process, Administration section, User Management % and General.

\subsection{Functional Requirements}
\subsection{Non-Functional Requirements}
\subsection{Constraints}
\subsubsection{Solution Constraints}
\subsubsection{Project Constraints}
\subsubsection{Process Constraints}