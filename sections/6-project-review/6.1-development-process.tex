As already mentioned in the introductory chapter of this report, we tried to lean the activities of our development process on the SCRUM process presented in module SWT-FSE-B.
In practice, we proceeded as follows:
\subsubsection{Sprint 0}
At the beginning - before the groups were even fixed - we had the opportunity to familiarize ourselves with the technologies used in the project, such as Spring Boot, Docker, DBeaver, and Scene Builder, as well as build our development environment.
Unfortunately, not everyone in the team managed to finish this before Sprint 0, so some catching up had to be done.
In Sprint 0, we therefore also had to take care of our development environment in addition to the Blast Off activities.
Nevertheless, everyone in the team was very motivated, and we were able to get off to a good start in the project.
    At the start of the project, we nominated our Product Owner as specified and also directly elected a Scrum Master to help us structure our activities and meetings.
We had initial difficulties especially with the planning of our activities, the time estimation, and the distribution of our tasks.
Although we roughly distributed the tasks, we had underestimated the respective effort of each task.
On the other hand, we had few work resources due to illness, lack of attendance and / or lack of knowledge of some team members.
However, this was quickly resolved after signaling it to the chair and our team received reinforcements towards the end of the sprint.

\subsubsection{Sprint 1}
    In Sprint 1, we then started our first increment after a Sprint Planning Meeting.
We concentrated on the first main functionality, the creation of offers.
In the Sprint Planning meeting, we were all relatively unprepared, which was shown by the unclear formulation of our tasks.
Our product owner was also a bit clumsy at the beginning due to lack of experience and was not able to prioritize the tasks accordingly.
That lead to a lot of merge conflicts, as the tasks and their output were often overlapping.
We reflected on all these aspects in the retrospective and were thus able to implement them better in Sprint 2.

\subsubsection{Sprint 2}
    In Sprint 2, after extensive planning, we decided to use Pair Programming more intensively to complete tasks more effectively.
This made sense to us mainly because we divided the tasks by functionality and not necessarily strictly by frontend / backend.
We may have been too focused on dividing and formulating the tasks "correctly" as we let the time estimation of the tasks take a bit of a back seat.
Another challenge for us was that we sometimes lost sight of our sprint goals, because new aspects / new requirements kept coming up, which we didn't take into account during the planning.
Here, too, we used the retrospective to reflect on the course of the sprint.

\subsubsection{Sprint 3}
    In sprint 3, the focus also laid on the features that were important to the customer.
After we got a better handle on the problems with task distribution, timeboxing and time estimation, we finally found a better workflow for us.
Nevertheless, there were some issues that still didn't work optimally: We had a lot of technical issues and error messages that cost a lot of time, and also the handling of merge requests was not effective.
This caused us to slow down in the development process, and again we did not reach our sprint goal.
As the client also remarked in the review meeting, this was largely due to our goal setting, which was too ambitious.
While we were able to show the client a working increment in each review meeting, we had still missed our sprint goals up to that point.

We took this into the next sprint.
\subsubsection{Sprint 4 and 5}
    Sprint 4 and Sprint 5 were sprints where we did an excellent job of overcoming the challenges of the previous sprints.
Our meetings were more structured, we were all well prepared, the tasks were clearly defined, we got them done in the estimated time, and we met our deadlines for the most part.
Our management on GitLab was also effective and we regularly added working and fully reviewed features to our increment.
In the last two sprints, we applied more quality assurance techniques (see the chapter on QA), which also meant that both our team and the customer seemed satisfied.
Towards the end, we had not met all the requirements, although most of them were optional, and we had also agreed on this with the customer.

\subsubsection{Process change and improvements}
All in all, and as already described, our way of working and thus our development process has always evolved to better.
In one sprint we improved the aspects that were previously perceived as challenging or led to problems.
For example: our task formulation and creation, our time management, our handling of merge requests as a review technique, and also the documentation of changing requirements.
But also our quality assurance had improved during the sprints, as it was too much limited to unit tests in the beginning.
The review meetings, in which we received a lot of feedback from the client on our development process, were particularly helpful for these improvements.
    What could be improved in our development process in general is to take the role of the Scrum Master more seriously as taught in SWT-FSE-B,
to refactor our code more regularly to keep it maintainable and simple, and to feel more responsible for every line of code in the group, even though we didn't write it.
The last points are practices of Extreme Programming (XP) as taught in SWT-FSE-B.
Other practices of XP that we could have used to improve our development process are the test-driven approach and 'sustainable pace'.
The latter is a principle to avoid overtime, as this reduces code quality and productivity.
Sometimes, we worked too long on the project and had to refactor code and change artefacts, because we weren't that productive.

\subsubsection{Comparison with SCRUM and other development processes}
Our development process was incremental: a mix consisting of SCRUM methods and Extreme Programming (XP) methods.
The roles (Scrum Team, Scrum Master, Product Owner), the meetings (Sprint Planning, MidSprint, Review, PO Meeting) and the artefacts (Product Backlog, Sprint Backlog) were based on SCRUM, whereby our Scrum Master had to pay less attention to ensuring that SCRUM practices were followed.
In practice, our Scrum Master was responsible for documenting the meetings in the Sprint.
Only in the first sprint, the Scrum Master ensured the SCRUM practices are respected.
After the first sprint, the remaining Scrum Master activities were taken over by the Product Owner in our team.
In addition, in the first two sprints, our Sprint Backlog was also our Product Backlog, which is also different from "the textbook".
This happened just because we weren't experienced enough to fill the product backlog in advance.
In terms of implementation techniques, we heavily used Pair Programming, which is a practice in the XP process and a good way to maintain better quality.
Also, selecting user stories and breaking them down to tasks to plan the next working increment was part of our iterations.
We also did Refactoring.
Not in every iteration, but in the moments when changing requirements demanded it.
Whenever we finished a task, we tried to review it and integrate it in our working increment as soon as possible.
Both the Refactoring and the continuous integration of code are practices of XP, too.
