If we could re-start the project, we would give more importance to the traceability of requirements and information in general.
We always made detailed meeting notes and therefore didn't lose information per se.
But on the other hand, it was sometimes hard to remember in which meeting notes to find what we were looking for.
It might have been helpful to extract the information right away, document it in the right place and therefore continuously update our artefacts.
Overall, our documentation represents a point that we could have improved in some places.
Often we were very focused on implementing the features and the focus on documenting was lost.
One example is updating our snowcards.
However, our constant verbal communication allowed us to avoid major issues due to lack of documentation.
Nevertheless, we notice the lack especially now when writing the report.
Also, for formulating and keeping our sprint goal, we would re-take our documentations more often for cross-checking.
Basically, we would take the quality assurance techniques more seriously from the very beginning, since we now understand that there are techniques like object lifecycle diagrams, prototypes and more that can be applied before even implementing some code.
That is one of the biggest lessons learned: Quality assurance doesn't only contain unit and integration tests.
What should stay the same if we re-started the project, is our team spirit and our communication.
In summary, we can say: It was an exciting and challenging project that made us realize to what extent software engineering is a socio-technical discipline.
No matter how well everyone can code: Without teamwork, communication and good project management, the result will not be optimal.