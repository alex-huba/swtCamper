On review meeting our PO held a short presentation for client and customer and showed the current state of SWTcamper. During the presentation PO showed all new functionality, namely (un)block provider or renter, booking history, operator influence, exclude renters. Also, she told about small fixes in the UI and backend. 

Generally speaking, our team achieved all what we planned on sprint planning meeting. So, sprint goal has been met.

Customer made comments about some of our decisions regarding frontend and overall features. 

Here is the list:

\begin{enumerate}
	\item Button “Registrieren” shall be straight under the button “login”.
	\item Shadows in card-layout are too bright and big
	\item Size of images shall have fixed size
	\item Search functionality in log-history is needed 
	\item Export log-history in external file would be a great feature
\end{enumerate}

\quad After that, client made a few remarks regarding our presentation, understanding the key principles of agile software development and upgrade-feature. Here is the list:
\begin{enumerate}
	\item The sprint goal wasn’t presented. Instead, there were a pile of use cases, which is not the same. 
	\item Software quality is not only about automated test. Use cases don’t relate to software quality.
	\item Why did you decide to implement upgrading functionality? It wasn’t in project brief.
	\item Don’t forget to synchronize with requirements and try always to get customer feedback before implementing a new feature.
\end{enumerate}

To sum up, customer and client were satisfied with the result of the fourth sprint.


Also, we learned that our team shall be more precise when we are trying to state
the sprint goal. We understood that use cases don’t relate to quality assurance because they only describe functionality. Also, we understood that it is not enough to implement unit and integration test in order to create high quality software. We have to stick to stakeholder’s requirements and not to create our own. 

Our team understood the mistakes from the third sprint and tried to avoid them in the current sprint. Most importantly, our team was more synchronized and that’s why it was less merge conflicts during the merge sessions. Pair programming was also very helpful because you never know how you colleague would solve the same issue. Because of this it is useful sometimes to solve some tasks together. 

Also, our team upgraded skill in creating issues and choosing the right amount of work for the sprint. At the first three sprints tasks were sometimes enormously big and complex to solve the within one third of the sprint. Our team chose a right time management strategy and we directly showed our working increment without making a production environment without some features, which cannot work normally. Documentation is important. The more meaningful docs you have, less problems in development process occur.