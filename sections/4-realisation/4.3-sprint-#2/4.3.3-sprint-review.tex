Upon starting the application, the home page is displayed. A search area offering several filtering options (Vehicle Type, Vehicle Brand, Construction Year, Max. price per day, Location, Engine, Transmission, Nr. of seats, Nr. of beds, Roof tent, Roof rack, Bike rack, Shower, Toilet, Kitchen, Fridge) can be seen in the upper half. Listed below that are all offers in the database, in textual form. After entering/choosing the desired filter options and clicking the green “search” button, only offers matching those options are displayed.
To the left side, a purple, vertical navigation bar can be seen. It contains three icons, a “hamburger menu” icon, which, when clicked, expands the navigation bar and reveals the names for the next two icons: Homepage and Login. Homepage is the default option when starting the application. When clicking Login, the user is taken to the Login page. She/he can either, if a user account exists, enter username and password and access the apps further features, or click the “sign up” link, which takes the user to the “Sign up” page. 7 mandatory text fields are displayed (Username, Password, Repeat Password, Email, Phone, Name, Surname). When entering the desired content into the fields, a warning message is displayed until the input requirements for that field are satisfied (e.g., Username needs at minimum 5 characters). Below that, one, two or all of three checkboxes (“I’m an operator”, “I’m a renter”, “I’m a provider”) can be checked. These determine the role of the user account (which as of yet has no effect, meaning the user sees all menu items regardless of chosen role). After all is said and done, the “sign up” button below becomes clickable and a message pops up, informing the user that her/his data will be checked by an operator shortly (this functionality is not yet implemented, meaning the user can already log in).

The basic functionalities of the user stories chosen for this sprint were implemented:

\begin{itemize}
    \item A user can log in by supplying the correct email and password
    \item A user can register a user account by supplying the necessary data
\end{itemize}

The User Story 01: Registration is finished. For the other stories, a number of acceptance criteria are still unfulfilled (e.g., different views for different user roles, certain filter options), without which the user stories are not counted as finished.

In the review meeting’s retrospective, the newly adapted pair programming technique was described as having worked out well, it was even stated that more pair programming is desired, as well as working together in general. Identified as a problem was the inability of the team to accurately estimate the time needed to complete a task. This led to tasks not getting finished. To remedy this, we planned to reduce the number of tasks moved into the sprint backlog at the beginning of a sprint. By this we would avoid having too much to do in the time given, and benefit from the psychological effects of meeting our goals. Also mentioned was the tendency to start coding at the beginning of a new sprint, without regards to acceptance criteria or other quality assurance measures. This has two types of consequences: Firstly, the subsequent “repairs” take a long time, as seen in this sprint with the task “Fix dev branch”, which resulted from insufficient planning of how to handle branches and merges. To fix this, the team decided to try to communicate more, and more effectively, so any foreseeable issues can be avoided before they become real problems. This includes talking about interfaces, classes and the general structure of the application early. Secondly, many new ideas/requirements/relevant aspects can be discovered while coding. Without the necessary discipline, one can lose track of the actual goals. Consequently, it was proposed to thoroughly document any newly discovered requirements, acceptance criteria and such, and to communicate them in the dailies. As we observed for now two sprints how tasks can, if it is not avoided, be carried into the next sprint, we emphasized the need for thorough planning, meaning in this case that the leftover tasks are finished first, so that the team can continue working from a solid foundation, so to speak. The problems arising from leftover tasks were felt during the time the issue “Fix dev branch” was not finished, were nobody was really sure about the current state of the application.
