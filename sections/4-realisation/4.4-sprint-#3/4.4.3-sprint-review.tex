Although we didn't meet all of our sprint goals again, quite a lot was achieved.
We finished all the leftover tasks from sprint 2, integrated the feedback from the last review meeting and implemented most functionalities of the booking process.

In particular we fixed minor bugs with the registration and login, redesigned a proper view for an offer and made a card layout for the search results.
These card layouts contain a thumbnail of the camper, some important information about it and a button to get to the before mentioned view.
When a user now opens the view of an offer, he has two calendar fields at the bottom where he can choose the date for his booking,
and a button to submit the booking request. Then a notification pops up which asks the user if he really wants to book the offer
and then the booking is done. The user could at this stage also view his booking history.
We also made progress on the unit tests for Offer- and Userservice although we were not able to reach the desired code coverage in this sprint.
What we couldn't finish was the booking process from the providers perspective, specifically a notification a provider should get and the possibility to review and approve/disapprove a booking.

The feedback from the customer was very positive, only some minor remarks about some color choices were given and that inactive offers shall not be seen by other users.

Prof. Luettgen also gave us some good tips on how to improve our presentation for the customer:
\begin{enumerate}
    \item more elements in the database for presenting the different use cases
    \item more precise planning of the use cases (with misuse cases, exceptions and test parameter)
    \item better formulation of the sprint goals
\end{enumerate}

In the retrospective we noticed that our teamwork got better since the last sprints although there were still some issues.
The smaller tasks helped a lot to get more done and structure the workload better but we were still to slow with the reviewing of merge requests, which led to another crunch session the day before the sprint review.
We also wasted some time figuring out errors alone that could have been solved faster with the team.
