In the third sprint we also used the agile SCRUM approach for developing.
In both sprint 1 and sprint 2 we had unfinished tasks and open merge requests that should have been included in the current sprint.
We also didn't meet all of our sprint goals, so in this sprint we specifically tried to improve our working processes.
Trying to follow the principle of continuous integration we decided to make smaller tasks that can be done faster and to merge them shortly after they were done.
Another aspect was that we didn't want to set our sprint goals too high this time to prevent increasing frustration for not achieving the planned progress.
In the previous sprints we also had the problem that some tasks were not completely planned through and discussed which led to some work being redundant, missing or not fitting to the other tasks.
Because of this we decided to assign two people to a task, which was very helpful since you always had another person you could ask for advice or share work with.
This also reduced the amount of time we spent in dailies since many question we had could be solved by the two people assigned and didn't have to be solved in the group.
The prototype we created earlier and showed to the customer was also very useful when implementing the Views for the project since we had a layout we could use for orientation which was already approved by the customer.

Artifacts produced in Sprint were the booking entity with its controller, service and interface to make a booking possible in the backend.
We also wanted to redesign our view of an offer because we were working with a rather rudimentary version before which just showed the necessary data and also because we changed some of the attributes of an offer and added the possibility to add pictures of the camper.
After  looking at the input validation of the registration and the offer creation we also decided to implement a validation helper class.
The reason behind this was that a lot of the checks we used eg. for the length or if it isn't empty were repeatedly used by different inputs.
So to minimize redundant code and also improve the maintainability of the input validation we bundled all the checks in the static validation helper class and just called the respective methods when they were needed.
We had some problems writing tests for our classes because of inexperience in writing tests but also because of technical difficulties some members had.
We solved this later by helping the members with the technical difficulties and asking for help from Kerstin and teaching the inexperienced members but it took time and got better in the following sprints.
